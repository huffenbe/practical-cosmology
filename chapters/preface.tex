\section*{Preface}

We are making vast improvements in our knowledge of the universe via the acquisition of precision data, and understand that we live in a vast universe about which we only understand the general outlines.  The normal matter that we understand through the standard model of particle physics makes up only 5 percent of the contents of the universe.  The dark matter and dark energy are distinct and their exotic properties are not well understood.  We study the brightness of supernovae, the fluctuations in the temperature and polarization of the cosmic microwave background, the clustering of galaxies, the abundance of primordial nuclei, and other observations.   Students and researchers working in this data cosmology should have a background in the theory underlying the cosmological models, and most cosmology textbooks focus on these theoretical aspects.  They also need a deep understanding of the observations and the statistics computed from the data to make comparisons to the theoretical models.  This book addresses this second need.  These tasks are somewhat different.  For example, the computation of the theoretical power spectrum of the cosmic microwave background requires a detailed and complicated computation using perturbation theory in general relativity, but the processing of a CMB measurement to get a data power spectrum to compare to theory requires no general relativity at all.  Indeed in this book, there is little general relativity past the first chapter, and not much there either.  Instead the reader will find probability, statistics, Monte Carlo simulations, harmonic transforms, filters, and signal processing techniques.  These are some of the practical tools for measuring the details of our universe.

