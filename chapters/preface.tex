\section*{Preface}

We are making vast improvements in our knowledge of the Universe via the acquisition of precision data, and understand that we live in a vast and complicated universe about which we only understand the general outlines.  The normal matter that we understand through the standard model of particle physics makes up only five percent of the contents of the universe.  The dark matter and the dark energy are distinctive substances; their  properties are exotic and not well understood.  We study the brightness of supernovae, the fluctuations in the temperature and polarization of the cosmic microwave background, the clustering of galaxies, the abundance of primordial nuclei, and other observations.   Students and researchers working in cosmology should have a background in the theory underlying the cosmological models, and most cosmology textbooks focus on these theoretical aspects.  They also need a deep understanding of the observations and the statistics computed from the data to make comparisons to the theoretical models.  This text addresses the second need.

The two tasks are somewhat different.  For example, the computation of the theoretical power spectrum of the cosmic microwave background requires a detailed and complicated computation using perturbation theory in general relativity, but the processing of a CMB measurement to get a data power spectrum to compare to that theory requires no general relativity at all.  Indeed, in this book, there is little general relativity past the first chapter, and not much there either.  Instead the reader will find probability, statistics, Monte Carlo sampling, harmonic transforms, filters, and signal processing techniques.  These are some of the practical tools for measuring the details of our Universe.

\section*{Acknowledgments}
I thank
Peter Brown
and
Nao Suzuki
for useful scientific discussions.
Student readers who helped find typographic errors and other problems include:
George Fagan,
Aaron Hughes,
Sophia Paulino Korte,
Mason Moenter,
and
Annabella Yang.
\textcolor{red}{More acknowledgments...}
