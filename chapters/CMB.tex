\chapter{The Cosmic Microwave Background}

What is the cosmic microwave background?

The CMB power spectrum is pillar of modern cosmology because it is both measureable and predictable from theory.

\section{Anistropies}
To examine correlations in the CMB we again want to decompose it into a series of harmonic functions.  Because the surface of the sphere has a finite area, the spectrum of harmonics is discrete.  We can write it as
\begin{equation}
  T(\theta,\phi) = \sum_{lm} a_{lm} Y_{lm}(\theta,\phi)
\end{equation}
where the $Y_{lm}$ functions are the spherical harmonics.  Most of the time, the exact form of these functions are not important, but we sometimes use some definitions and identities that are listed in a supplemental section at the end of this chapter.





\section{Signal, beam, and noise}


\section{Measuring the power spectrum}

including cross spectra


\section{Mapmaking}


\section{The flat-sky approximation}


\section{Polarization anisotropies}



\section*{Supplemental: spherical harmonic functions}

The function is band limited if there are only zero harmonics above some $l_{\rm max}$.  The CMB is not band limited, but our observation of the CMB is in practice band limited by the finite resolution of the telescope we use to observe it.

The spherical harmonics are complex functions
\begin{equation}
  Y_{lm}(\theta,\phi) = (-1)^m \sqrt{\frac{2 l + 1}{4\pi} \frac{l-m}{l+m}} P_l^m(\cos \theta) \exp(im\phi)
\end{equation}
that depend on the associated Legendre polynomials $P_l^m$.  In cosmology we use the same conventions that are common in quantum mechanics, in particular placing the factor of $(-1)^m$ (``Condon-Shortley'' phase) in the definition of the spherical harmonics and not in the definition of the associated Legendra polynomicals.

These functions have several important identities.  They are orthonormal functions integrated over the on the whole sphere
\begin{equation}
  \int d\Omega Y_l^m (\theta,\phi) Y_{l'}^{m'}(\theta,\phi) = \delta_{ll'} \delta_{mm'}
\end{equation}
This in particular means that for $l=0$, $m=0$
\begin{equation}
  Y_{00} = \frac{1}{(4\pi)^{1/2}}
\end{equation}
