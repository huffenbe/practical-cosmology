\chapter*{Introduction}

The universe is vast and old and dynamic.  We know that the time since the Big Bang may be $\sim 13.8$ billion years, although we are a few percent uncertain because we are still trying to understand the Universe's present-day expansion rate, and our knowledge of the age depends on that.  A photon or graitational wave traveling at light speed from the time of the Big Bang and just arriving here now would have come from a point in space that is today 46.3 billion light years away.  This number is bigger than 13.8 billion light years because the Universe has expanded in the meantime.  

We have some remarkable facts to start with.
\begin{enumerate}
  \item On large scales, $>1$ gigaparsecs (Gpc), the universe at the present day consists of a uniform sea of galaxies, stretching in all directions.
  \item On average, these galaxies move apart (as observed in Doppler-shifted light).  We interpret this as the whole spacetime expanding.
  \item Under the rules of general relativity, we can model the {scale factor} of the universe as a function of time, telling us how the spacetime.  Amazingly, the scale factor approaches zero a finite time ago, which is what we mean by the time of the Big Bang.
\end{enumerate}

Noting that the universe appears to be \textbf{homogeneous} (uniformly the same in every location) and \textbf{isotropic} (the same in every direction), we assume the ``cosmological principle'' that we are not at a special location.  This we mean in a statistical sense; the Earth is different from a star or the vacuum of outer space, but on a similar planet in a similar galaxy, a similar creature could study cosmology, and that creature would draw the same conclusions as us.  Because the universe is homogeneous and changing in time, we first need to study uniform but dynamic spacetimes.

\textcolor{red}{Describe the contents of each chapter in turn}
